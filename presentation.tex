\documentclass{beamer}
\usetheme[pageofpages=of,% String used between the current page and the
                         % total page count.
          bullet=circle,% Use circles instead of squares for bullets.
          titleline=true,% Show a line below the frame title.
          alternativetitlepage=true,% Use the fancy title page.
       %   titlepagelogo=logo-polito,% Logo for the first page.
       %   watermark=watermark-polito,% Watermark used in every page.
       %   watermarkheight=100px,% Height of the watermark.
       %   watermarkheightmult=4,% The watermark image is 4 times bigger
                                % than watermarkheight.
          ]{Torino}

\author{Brendon J. Brewer}
\title{Bayesian probability theory}
\institute{Department of Statistics, The University of Auckland}
\date{{\color{blue} https://www.stat.auckland.ac.nz/\~{ }brewer}\\
\vspace{10pt}
{\color{blue} @brendonbrewer}}

\begin{document}

\begin{frame}[t,plain]
\titlepage
\end{frame}

\begin{frame}[t]{Propositions}
Propositions are {\it statements that can be either true or false}.\\
Examples:\vspace{20pt}

$A$: {\small Hillary Clinton will be president of the USA on June 14th, 2017.}\\
$B$: {\small The current temperature in Auckland is greater than 14$^{\circ}$C.}\\
$C$: {\small $\Omega_\Lambda \neq 0$.}\\
$D$: {\small David Hogg has had more than 500 mg of caffeine today.}
\end{frame}

\begin{frame}[t]{Propositions}
Propositions can be {\it combined} to make others, using the operators
{\bf and} ($\wedge$), {\bf or} ($\vee$), and {\bf not} ($\neg$): \vspace{20pt}

$A \wedge B$\\
$C \vee A$\\
$\neg D$\\
$\neg (A \vee (C \wedge D))$
\end{frame}

\begin{frame}[t]{Quantifying propositions}
Science is largely concerned with the {\it plausibility} of propositions.
Plausibility depends on the information you have, so is a function of {\it
two} propositions. Examples:\vspace{20pt}

$P(A | B)$\\
$P(\neg D | (C \vee D))$\\
$P((C \wedge D) | \neg A)$\vspace{20pt}

Read the ``$|$'' as ``{\bf given}'' or ``{\bf conditional on}''.
\end{frame}


\begin{frame}[t]{Constraints on plausibility}
Some properties of plausibility:\vspace{20pt}

$P(A \vee B) \geq P(A)$\\
%$P(A \wedge B) \leq P(A)$\\
$P(A \vee (B \vee C)) = P((A \vee B) \vee C)$\vspace{20pt}

Where'd the RHS go? $|| D$
\end{frame}

\begin{frame}[t]{Constraints on plausibility}
Some properties of plausibility:\vspace{20pt}

$P(A \vee B) \geq P(A)$\\
$P(A \wedge B) \leq P(A)$\\
$P(A \vee (B \vee C)) = P((A \vee B) \vee C)$\vspace{20pt}

These imply the {\bf sum rule} and the {\bf product rule}, and hence that
{\bf plausibilities are probabilities}.
\end{frame}

\begin{frame}[t]{Sum and product rule\hspace{170pt}$|| C$}
$P(A \vee B) = P(A) + P(B) - P(A \wedge B)$\\
\vspace{20pt}
$P(A \wedge B) = P(A)P(B|A)$
\end{frame}

\begin{frame}[t]{Bayes' Rule\hspace{170pt}$|| I$}
Bayes' Rule (consequence of product rule and commutativity of {\bf and})
\begin{equation}
P(H | D) = \frac{P(H)P(D|H)}{P(D)}
\end{equation}
where\vspace{20pt}
$P(D) = P(H)P(D | H) + P(\neg H)P(D | \neg H)$
(consequence of sum rule)
\end{frame}

\begin{frame}[t]{Terminology\hspace{170pt}$|| I$}
\begin{equation}
P(H | D) = \frac{P(H)P(D|H)}{P(D)}
\end{equation}
\begin{equation}
\textnormal{(posterior probability)} =
\frac{\textnormal{(prior probability)}\times\textnormal{(likelihood)}}{\textnormal{(marginal likelihood)}}
\end{equation}
\end{frame}

\begin{frame}[t]{Another Bayes' Rule\hspace{170pt}$|| I$}
For $N$ mutually exclusive, exhaustive hypotheses
$H_1, H_2, ..., H_N$, we have $N$ posterior probabilities:

\begin{equation}
P(H_i | D) = \frac{P(H_i)P(D|H_i)}{P(D)}
\end{equation}

where
\begin{equation}
P(D) = \sum_{i=1}^N P(H_i)P(D|H_i)
\end{equation}
\end{frame}

\begin{frame}[t]{Another Bayes' Rule\hspace{170pt}$|| I$}

{\bf This is the most important form of Bayes' rule.}

\begin{equation}
P(H_i | D) = \frac{P(H_i)P(D|H_i)}{P(D)}
\end{equation}

where
\begin{equation}
P(D) = \sum_{i=1}^N P(H_i)P(D|H_i)
\end{equation}
\end{frame}


\begin{frame}[t]{Exercises!\hspace{170pt}$|| I$}
\end{frame}

\begin{frame}[t]{Parameter estimation}
In most applications, we can use the ``parameter estimation'' story. E.g.
Let $\theta$ be a quantity we want to know. Then the hypotheses might be:

\begin{eqnarray}
H_1 &\equiv& \theta = 5\\
H_2 &\equiv& \theta = 6\\
H_3 &\equiv& \theta = 7
\end{eqnarray}

The data could also be a number. E.g. the conditioning proposition could be
\begin{equation}
D = 4
\end{equation}

The previous version of Bayes' rule can be applied to each of the.
\end{frame}

\begin{frame}[t]{Parameter estimation}
Define the prior distribution $p(\theta)$


\begin{equation}
p(\theta | D) \propto p(\theta)p(D|\theta)
\end{equation}

This notation hides {\bf a lot}!
\end{frame}



\end{document}

