\documentclass{beamer}
\usetheme[pageofpages=of,% String used between the current page and the
                         % total page count.
          bullet=circle,% Use circles instead of squares for bullets.
          titleline=true,% Show a line below the frame title.
          alternativetitlepage=true,% Use the fancy title page.
       %   titlepagelogo=logo-polito,% Logo for the first page.
       %   watermark=watermark-polito,% Watermark used in every page.
       %   watermarkheight=100px,% Height of the watermark.
       %   watermarkheightmult=4,% The watermark image is 4 times bigger
                                % than watermarkheight.
          ]{Torino}

\author{Brendon J. Brewer}
\title{Bayesian probability theory}
\institute{Department of Statistics, The University of Auckland}
\date{{\color{blue} https://www.stat.auckland.ac.nz/\~{ }brewer}\\
\vspace{10pt}
{\color{blue} @brendonbrewer}}

\begin{document}

\begin{frame}[t,plain]
\titlepage
\end{frame}

\begin{frame}[t]{Propositions}
Propositions are {\it statements that can be either true or false}.\\
Examples:\vspace{20pt}

$A$: {\small Hillary Clinton will be president of the USA on June 14th, 2017.}\\
$B$: {\small The current temperature in Auckland is greater than 14$^{\circ}$C.}\\
$C$: {\small $\Omega_\Lambda \neq 0$.}\\
$D$: {\small David Hogg has had more than 500 mg of caffeine today.}
\end{frame}

\begin{frame}[t]{Propositions}
Propositions can be {\it combined} to make others, using the operators
{\bf and}, {\bf or}, and {\bf not}: \vspace{20pt}

$A \wedge B$\\
$C \vee A$\\
$\neg D$\\
$\neg (A \vee (C \wedge D))$
\end{frame}

\begin{frame}[t]{Quantifying propositions}
Science is largely concerned with the {\it plausibility} of propositions.
Plausibility depends on the information you have, so is a function of {\it
two} propositions. Examples:\vspace{20pt}

$P(A | B)$\\
$P(\neg D | (C \vee D))$\\
$P((C \wedge D) | \neg A)$\vspace{20pt}

Read the ``$|$'' as ``{\bf given}'' or ``{\bf conditional on}''.
\end{frame}


\begin{frame}[t]{Constraints on plausibility}
Some properties of plausibility:\vspace{20pt}

$P(A \vee B) \geq P(A)$\\
%$P(A \wedge B) \leq P(A)$\\
$P(A \vee (B \vee C)) = P((A \vee B) \vee C)$\vspace{20pt}

Where'd the RHS go? $|| D$
\end{frame}

\begin{frame}[t]{Constraints on plausibility}
Some properties of plausibility:\vspace{20pt}

$P(A \vee B) \geq P(A)$\\
%$P(A \wedge B) \leq P(A)$\\
$P(A \vee (B \vee C)) = P((A \vee B) \vee C)$\vspace{20pt}

These imply the {\bf sum rule}.
\end{frame}

\begin{frame}[t]{Sum rule}
$P(A \vee B) = P(A) + P(B) - P(A \wedge B)$
\end{frame}

\end{document}

